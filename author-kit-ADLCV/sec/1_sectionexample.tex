\section{Section 1 - IMDB Transformers Encoder}

This is an example of a section structure for a weekly exercise.

\subsection{Introduction}
\lipsum[6]

\subsection{Models}
\lipsum[8]

\subsection{Datasets}
\lipsum[4]

\subsection{Experiments and Results}

\lipsum[8]

\subsection{Discussion}

\lipsum[4]

\begin{table}
  \centering
  \caption{\label{tab:example}%
    Example of Standard Table (caption at the top).
  }
  \begin{tabular}{lc}
    \toprule
    Method & Frobnability \\
    \midrule
    Theirs & Frumpy \\
    Yours & Frobbly \\
    Ours & Makes one's heart Frob\\
    \bottomrule
  \end{tabular}
\end{table}

\begin{figure}[t]
  \centering
  \fbox{\rule{0pt}{2in} \rule{0.9\linewidth}{0pt}}
   %\includegraphics[width=0.8\linewidth]{egfigure.eps}

   \caption{Example of figure with caption.}
   \label{fig:onecol}
\end{figure}


\begin{figure*}
  \centering
  \begin{subfigure}{0.68\linewidth}
    \fbox{\rule{0pt}{2in} \rule{.9\linewidth}{0pt}}
    \caption{An example of a subfigure.}
    \label{fig:short-a}
  \end{subfigure}
  \hfill
  \begin{subfigure}{0.28\linewidth}
    \fbox{\rule{0pt}{2in} \rule{.9\linewidth}{0pt}}
    \caption{Another example of a subfigure.}
    \label{fig:short-b}
  \end{subfigure}
  \caption{Example larger figures, placed in a new page with \texttt{\textbackslash begin\{figure*\}}.}
  \label{fig:short}
\end{figure*}